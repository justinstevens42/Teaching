%Author:  Justin Stevens
%Last Modified:  August 2nd, 2016
\documentclass[12pt,openany]{book}
\setlength{\headheight}{15pt}
\usepackage{amsmath, amsthm, amssymb}
\usepackage{mdframed}
\usepackage{lipsum}
\newmdtheoremenv{thm}{Theorem}[section]
\newmdtheoremenv{exmp}{Example}[section]
\theoremstyle{definition}
\newtheorem{defi}{Definition}[section]
\newenvironment{soln}{\begin{proof}[Solution]}{\end{proof}}
\newtheorem{prob}{Problem}[section]
\newtheorem*{comment}{Comment}
\setcounter{chapter}{1}
\setcounter{section}{0}
\usepackage{titlesec}
\renewcommand*\thesection{\arabic{section}}  %important  
\usepackage{cancel}
\usepackage[margin=3cm]{geometry}
\usepackage{hyperref}
\usepackage{fancyhdr}
\pagestyle{fancy}
\fancyhead{}
\fancyfoot{}
\theoremstyle{definition}
\newtheorem*{case}{Example}
\rhead{Page \thepage}
\newenvironment{dedication}
    {\vspace{6ex}\begin{quotation}\begin{center}\begin{em}}
    {\par\end{em}\end{center}\end{quotation}}
\newcommand{\HRule}{\rule{\linewidth}{0.5mm}} % Defines a new command for the horizontal lines, change thickness here

%----------------------------------------------------------------------------------------
%	PRESENTATION INFORMATION
%----------------------------------------------------------------------------------------

\newcommand*{\mytitle}{A-Star 2016 Winter Math Camp } % Title
\newcommand*{\runninghead}{AMC Number Theory} % Running head displayed on almost all slides
\newcommand*{\myauthor}{Justin Stevens} % Presenters name(s)
\newcommand*{\mydate}{\formatdate{26}{12}{2016}} % Presentation date
\newcommand*{\myuni}{A-Star 2016 Winter Math Camp} % University or department

%----------------------------------------------------------------------------------------

\begin{document}
	

%----------------------------------------------------------------------------------------
%	TITLE SLIDE
%----------------------------------------------------------------------------------------

% Title slide - you may have to tweak a few of the numbers if you wish to make changes to the layout
\thispagestyle{empty} % No slide header and footer
\begin{tikzpicture}[remember picture,overlay] % Background box
\node [xshift=\paperwidth/2,yshift=\paperheight/2] at (current page.south west)[rectangle,fill,inner sep=0pt,minimum width=\paperwidth,minimum height=\paperheight/3,top color=mygreen,bottom color=mygreen]{}; % Change the height of the box, its colors and position on the page here
\end{tikzpicture}
% Text within the box
\begin{flushright}
\vspace{0.6cm}
\color{white}\sffamily
{\bfseries\Large\mytitle\par} % Title
\vspace{0.5cm}
\normalsize
\myauthor\par % Author name
\mydate\par % Date
\vfill
\end{flushright}



\clearpage


%----------------------------------------------------------------------------------------
%	TABLE OF CONTENTS
%----------------------------------------------------------------------------------------

\thispagestyle{empty} % No slide header and footer

\small\tableofcontents % Change the font size and print the table of contents - it may be useful to shrink the font size further if the presentation is full of sections
% To exclude sections/subsections from the table of contents, put an asterisk after \(sub)section like so: \section*{Section Name}

\clearpage

\section{Algebraic Manipulation}

In this section, we will explore several of my favourite problems involving algebraic manipulations.

\mybox{0.8\textwidth}{\begin{prob}[2000 AMC 12]  If $x,y,$ and $z$ are positive numbers satisfying $$x+\frac{1}{y}=4\:, y+\frac{1}{z}=1\:, \text{and}\: z+\frac{1}{x}=\frac73,$$ find the value of $xyz$. \end{prob} \begin{prob}[AoPS Introduction to Algebra] Let $A=x+\frac{1}{x}$ and $B=x^2+\frac{1}{x^2}$. Note that $(x+\frac{1}{x})^2=x^2+2+\frac{1}{x^2}$, therefore, $B=A^2-2$. Find formulas for $$C=x^3+\frac{1}{x^3}\:, D=x^4+\frac{1}{x^4}, E=x^5+\frac{1}{x^5}$$ in terms of $A$.  \end{prob}}

\subsection{2000 AMC 12}
\begin{proof}[Solution]
In order to get the $xyz$ term, we are motivated to multiply the $3$ equations together: \begin{eqnarray*} \left(x+\frac{1}{y}\right)\left(y+\frac{1}{z}\right)\left(z+\frac{1}{x}\right)&=&xyz+\frac{1}{xyz}+\left(x+y+z\right)+\left(\frac{1}{x}+\frac{1}{y}+\frac{1}{z}\right) \\ &=& \left(4\right)\left(1\right)\left(\frac73\right)=\frac{28}{3}. \end{eqnarray*}  

What can we do now to simplify this further?
\clearpage
We also add all $3$ of the equations: \begin{eqnarray*} \left(x+\frac{1}{y}\right)+\left(y+\frac{1}{z}\right)+\left(z+\frac{1}{x}\right)&=&4+1+\frac{7}{3}=\frac{22}{3} \\ &=& \left(x+y+z\right)+\left(\frac{1}{x}+\frac{1}{y}+\frac{1}{z}\right). \end{eqnarray*}  
Therefore, plugging this in to the first equation gives \begin{eqnarray*} xyz+\frac{1}{xyz}+\left(x+y+z\right)+\left(\frac{1}{x}+\frac{1}{y}+\frac{1}{z}\right) &=& xyz+\frac{1}{xyz}+\frac{22}{3} \\ &=&\frac{28}{3} \\ \implies \color{blue} xyz+\frac{1}{xyz} &=& \color{blue} 2. \end{eqnarray*}  

What's $xyz$ equal to then?

\clearpage

Multiply the equation through by $xyz$ and simplify: $$(xyz)^2+1=2xyz\implies (xyz)^2-2\cdot xyz+1=(xyz-1)^2=0.$$  
Therefore, $xyz=\boxed{1}$.  \end{proof}

\subsection{Exponent Mayhem} 

\begin{proof}[Solution]
	
We begin with $C$. Inspired by our method for computing $B$, we attempt to cube $x+\frac{1}{x}$: \begin{eqnarray*} (x+\frac{1}{x})^3&=&x^3+3\cdot x^2\cdot \frac{1}{x}+3\cdot x\cdot \frac{1}{x^2}+\frac{1}{x^3} \\ &=& x^3+3x+\frac{3}{x}+\frac{1}{x^3}. \end{eqnarray*} 
\clearpage

Therefore, \begin{eqnarray*} A^3&=&x^3+3\left(x+\frac{1}{x}\right)+\frac{1}{x^3} \:=\:C+3A \\ \implies \color{blue} C &=& \color{blue} A^3-3A. \end{eqnarray*}

There are two methods for finding $D$. One of them involves taking $x+\frac{1}{x}$ to the fourth power. In order to continue with this method, however, I must introduce the binomial theorem and Pascal's triangle.  

\end{proof}


\end{document}  