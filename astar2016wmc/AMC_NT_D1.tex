%Author:  Justin Stevens
%Last Modified:  August 2nd, 2016
\documentclass[12pt,openany]{book}
\setlength{\headheight}{15pt}
\usepackage{amsmath, amsthm, amssymb}
\usepackage{mdframed}
\usepackage{lipsum}
\newmdtheoremenv{thm}{Theorem}[section]
\newmdtheoremenv{exmp}{Example}[section]
\theoremstyle{definition}
\newtheorem{defi}{Definition}[section]
\newenvironment{soln}{\begin{proof}[Solution]}{\end{proof}}
\newtheorem{prob}{Problem}[section]
\newtheorem*{comment}{Comment}
\setcounter{chapter}{1}
\setcounter{section}{0}
\usepackage{titlesec}
\renewcommand*\thesection{\arabic{section}}  %important  
\usepackage{cancel}
\usepackage[margin=3cm]{geometry}
\usepackage{hyperref}
\usepackage{fancyhdr}
\pagestyle{fancy}
\fancyhead{}
\fancyfoot{}
\theoremstyle{definition}
\newtheorem*{case}{Example}
\rhead{Page \thepage}
\newenvironment{dedication}
    {\vspace{6ex}\begin{quotation}\begin{center}\begin{em}}
    {\par\end{em}\end{center}\end{quotation}}
\newcommand{\HRule}{\rule{\linewidth}{0.5mm}} % Defines a new command for the horizontal lines, change thickness here

%----------------------------------------------------------------------------------------
%	PRESENTATION INFORMATION
%----------------------------------------------------------------------------------------

\newcommand*{\mytitle}{A-Star 2016 Winter Math Camp } % Title
\newcommand*{\runninghead}{AMC Number Theory Day 1} % Running head displayed on almost all slides
\newcommand*{\myauthor}{Justin Stevens} % Presenters name(s)
\newcommand*{\mydate}{\formatdate{26}{12}{2016}} % Presentation date
\newcommand*{\myuni}{A-Star 2016 Winter Math Camp} % University or department

%----------------------------------------------------------------------------------------

\begin{document}
	

%----------------------------------------------------------------------------------------
%	TITLE SLIDE
%----------------------------------------------------------------------------------------

% Title slide - you may have to tweak a few of the numbers if you wish to make changes to the layout
\thispagestyle{empty} % No slide header and footer
\begin{tikzpicture}[remember picture,overlay] % Background box
\node [xshift=\paperwidth/2,yshift=\paperheight/2] at (current page.south west)[rectangle,fill,inner sep=0pt,minimum width=\paperwidth,minimum height=\paperheight/3,top color=mygreen,bottom color=mygreen]{}; % Change the height of the box, its colors and position on the page here
\end{tikzpicture}
% Text within the box
\begin{flushright}
\vspace{0.6cm}
\color{white}\sffamily
{\bfseries\Large\mytitle\par} % Title
\vspace{0.5cm}
\normalsize
\myauthor\par % Author name
\mydate\par % Date
\vfill
\end{flushright}

\clearpage

%----------------------------------------------------------------------------------------
%	TABLE OF CONTENTS
%----------------------------------------------------------------------------------------

\thispagestyle{empty} % No slide header and footer

\small\tableofcontents % Change the font size and print the table of contents - it may be useful to shrink the font size further if the presentation is full of sections
% To exclude sections/subsections from the table of contents, put an asterisk after \(sub)section like so: \section*{Section Name}

\clearpage

%----------------------------------------------------------------------------------------
%	PRESENTATION SLIDES
%----------------------------------------------------------------------------------------

\section{Introduction}

Welcome to A-Star Winter Math Camp 2016!  This is my fourth A-Star camp.  
\begin{itemize}
	\item  I've attended once as a student before.
	\item  I've taught the AMC class twice before in the summer of 2015 and 2016.
	\item Number Theory is my favourite subject to teach :).   
\end{itemize}


\clearpage

%------------------------------------------------

\subsection{Schedule}

\begin{table}[h]
	\centering
	\begin{tabular}{l l}
		\toprule
		\textbf{Time} & \textbf{Subject} \\
		\midrule
		9-10:30 AM & Number Theory \\
		10:45AM-12:15PM & Algebra \\ 
		1:45-3:15PM & Geometry \\
		3:30-5:00PM & Counting \\
		\bottomrule
	\end{tabular}
	\caption{A-Star Teaching Schedule}
\end{table}

\clearpage 
\subsection{Icebreaker Activity}

\begin{figure}[h]
	\centering\includegraphics[width=0.5\linewidth]{images/icebreaker.jpg}
\end{figure}

\clearpage

\subsection*{Three Truths and a Lie}

Write down three truths and one lie about yourself on your piece of paper.  I'll guess which one is the lie!  
Good luck guessing which one is my lie.

\begin{itemize}
	\item I've seen over 100 different bands live in concert.
	\item I've programmed a human sized robot.  
	\item My family has 2 cats.  
	\item I've competed in and won a crib race.  
\end{itemize}

\clearpage

\subsection*{Concerts:  \color{green} Truth}

\begin{figure}[h]
	\centering\includegraphics[width=0.23\linewidth]{images/concert1.jpg}
\end{figure}

\clearpage

\subsection*{Robot:  \color{green}  Truth}

\begin{figure}[h]
	\centering\includegraphics[width=0.4\linewidth]{images/robot.jpg}
\end{figure}

\clearpage

\subsection*{Cats:  \color{red}  (Deceptive) Lie!}

\begin{figure}[h]
	\centering\includegraphics[width=0.4\linewidth]{images/lie.jpg}
\end{figure}

\clearpage

\subsection*{We have 5...}

\begin{figure}[h]
	\centering\includegraphics[width=0.5\linewidth]{images/creampuff.jpg}
\end{figure}

\clearpage

\begin{figure}[h]
	\centering\includegraphics[width=0.6\linewidth]{images/kittens.jpg}
\end{figure}

\clearpage

%different picture of cupcake?
\begin{figure}[h]
	\centering\includegraphics[width=0.6\linewidth]{images/cupcake.jpg}
\end{figure}

\clearpage 
\subsection*{Crib Race??:  \color{green}  Truth}

\begin{figure}[h]
	\centering\includegraphics[width=0.32\linewidth]{images/crib.jpg}
\end{figure}

\clearpage
\subsection*{Celebration!}

\begin{figure}[h]
	\centering\includegraphics[width=0.32\linewidth]{images/crib2.jpg}
\end{figure}

\clearpage  

\section{Math Time}

The topic for today is divisibility and prime factorization.  %more intro to the topic and its importance  

\clearpage

\subsection{Divisibility Rules}

\begin{itemize}
\item  2 - Last digit is even.
\item  3 - Sum of the digits is divisible by 3.
\item 4  - Number formed by last two digits is divisible by 4.
\item 5 - Last digit is either $0$ or $5$.
\item 6 -  Divisibility rules for both $2$ and $3$ hold.
\item 7  - Take the last digit of the number and double it.  Subtract this from the rest of the number.  Repeat the process if necessary.  Check to see if the final number obtained is divisible by $7$.
\end{itemize}


\clearpage

\subsection*{Lucky Seven}

Choose \textbf{one} number below and determine if it is divisible by $7$!
\begin{itemize}
	\item $1729$
	\item $2,718,281$
	\item $16,180,339$
	\item $31,415,926,535$
\end{itemize}

\subsection*{Taxicab Number}

``It is a very interesting number; it is the smallest number expressible as the sum of two positive cubes in two different ways." -  Srinivasa Ramanujan (1919)
\begin{eqnarray*}
	1729 &\to& 172-2\cdot 9=154 \\ 
	154 &\to& 15-2\cdot 4=7
\end{eqnarray*}
Therefore, $1729$ \textbf{is} divisible by $7$.


\clearpage

\subsection*{Euler's Number}

\begin{eqnarray*}  
	2718281 &\to& 271828-2\cdot 1=271826 \\ 
	271826 &\to& 27182-2\cdot 6=27170 \\
	27170 &\to& 2717-2\cdot 0=2717 \\ 2717 &\to& 271-2\cdot 7=257 \\ 257 &\to& 25-2\cdot 7=11
\end{eqnarray*}

Therefore, $2718281$ is \textbf{not} divisible by $7$.

More on Euler's number ($e$) during Algebra lectures!  

\clearpage

\subsection*{The Golden Ratio - $\phi=\frac{1+\sqrt{5}}{2}=1.6180339\cdots$}

\begin{eqnarray*}
	16180339 &\to& 1618033-2\cdot 9=1618015 \\ 1618015 &\to& 161801-2\cdot 5=161791 \\ 161791 &\to& 16179-2\cdot 1=16177 \\ 16177 &\to& 1617-2\cdot 7=1603 \\ 1603 &\to& 160-2\cdot 3=154 \\ 154 &\to& 15-2\cdot 4=7
\end{eqnarray*}  

Hence, $16180339$ \textbf{is} divisible by $7$.

\clearpage

\subsection*{Pi}

$31,415,926,535$ is too big of a number.  Therefore, I wrote a computer program!



\begin{figure}[h]
	\centering\includegraphics[width=1\linewidth]{images/seven.png}
\end{figure}
\clearpage


\subsection{Bullet Points and Numbered Lists}

\begin{itemize}
\item Lorem ipsum dolor sit amet, consectetur adipiscing elit
\item Aliquam blandit faucibus nisi, sit amet dapibus enim tempus eu
\end{itemize}

\begin{enumerate}
\item Nulla commodo, erat quis gravida posuere, elit lacus lobortis est, quis porttitor odio mauris at libero
\item Nam cursus est eget velit posuere pellentesque
\item Vestibulum faucibus velit a augue condimentum quis convallis nulla gravida
\end{enumerate}

\clearpage

%------------------------------------------------

\subsection{Verbatim}

How to include a theorem in this presentation:
\begin{verbatim}
\mybox{0.8\textwidth}{
\begin{theorem}[Murphy (1949)]
Anything that can go wrong, will go wrong.
\end{theorem}
}
\end{verbatim}

\clearpage

%------------------------------------------------


\section{Displaying Information}

\clearpage

%------------------------------------------------

\subsection{Table}

\begin{table}[h]
\centering
\begin{tabular}{l l l}
\toprule
\textbf{Treatments} & \textbf{Response 1} & \textbf{Response 2}\\
\midrule
Treatment 1 & 0.0003262 & 0.562 \\
Treatment 2 & 0.0015681 & 0.910 \\
Treatment 3 & 0.0009271 & 0.296 \\
\bottomrule
\end{tabular}
\caption{Table caption}
\end{table}

\clearpage

%------------------------------------------------

\subsection{Figure}



\clearpage

%------------------------------------------------

\subsection{Theorem}

The most common definition of \mygreen{Murphy's Law} is as follows.

\mybox{0.8\textwidth}{ % Example of encapsulating text in a colored box
\begin{theorem}[Murphy (1949)]
Anything that can go wrong, will go wrong.
\end{theorem}
}

\begin{proof}
A special case of this theorem is proven in the textbook.
\end{proof}

\begin{remark}
This is a remark.
\end{remark}

\begin{algorithm}
This is an algorithm.
\end{algorithm}

\clearpage

%------------------------------------------------

\section{Citations}

An example of the \texttt{\textbackslash cite} command to cite within the presentation:

This statement requires citation \cite{Smith:2012qr}.

\clearpage

%------------------------------------------------

\thispagestyle{empty} % No slide header and footer

\bibliographystyle{unsrt}
\bibliography{sample}

\clearpage

%------------------------------------------------

\thispagestyle{empty} % No slide header and footer

\begin{tikzpicture}[remember picture,overlay] % Background box
\node [xshift=\paperwidth/2,yshift=\paperheight/2] at (current page.south west)[rectangle,fill,inner sep=0pt,minimum width=\paperwidth,minimum height=\paperheight/3,top color=mygreen,bottom color=mygreen]{}; % Change the height of the box, its colors and position on the page here
\end{tikzpicture}
% Text within the box
\begin{flushright}
\vspace{0.6cm}
\color{white}\sffamily
{\bfseries\LARGE Questions?\par} % Request for questions text
\vfill
\end{flushright}

%----------------------------------------------------------------------------------------

\end{document}