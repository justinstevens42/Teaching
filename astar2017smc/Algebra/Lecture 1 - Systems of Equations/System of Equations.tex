% !TeX document-id = {9dc4049a-302f-49cf-bd6d-ab733465c3ce}
% !TeX TXS-program:compile = txs:///pdflatex/[--shell-escape]
\documentclass[xcolor=dvipsnames, fontsize=11pt, % Font size
pagesize, % Write page size to dvi or pdf
parskip=half-, t]{beamer}

%
% Choose how your presentation looks.
%
% For more themes, color themes and font themes, see:
% http://deic.uab.es/~iblanes/beamer_gallery/index_by_theme.html
%
\mode<presentation>
  \usetheme{Madrid}      % or try Darmstadt, Madrid, Warsaw, ...
  \usecolortheme{beaver} % or try albatross, beaver, crane, ...
  %\usefonttheme{serif}  % or try serif, structurebold, ...
  \setbeamertemplate{navigation symbols}{}
  \setbeamertemplate{caption}[numbered]
  
\newcounter{example}
\newenvironment<>{exmp}[1][]{%
    \refstepcounter{example}\par\medskip
  \alert{\textbf{Example~\theexample.} }}{}
  
\newenvironment<>{thm}[1][]{%
 \par\medskip
\textbf{\textcolor{MidnightBlue}{\sffamily Theorem.  }}}{}
\newenvironment<>{exmp*}[1][]{%
\par\medskip
  \alert{\textbf{Example.} }}{}
  
\makeatletter
\newenvironment<>{myproof}[1][\proofname]{%
  \par
  \def\insertproofname{#1.}%
  \pushQED{\qed}
  \textit{\insertproofname}  }
{}
\makeatother

\setbeamertemplate{headline}{}
\setbeamercovered{transparent}

\usepackage{etoolbox}
\makeatletter
\patchcmd{\beamer@continueautobreak}{\frametitle}{\beamer@gobbleoptional}{}{\errmessage{failed to patch}}
\patchcmd{\beamer@continueautobreak}{\framesubtitle}{\beamer@gobbleoptional}{}{\errmessage{failed to patch}}
\makeatother

\makeatother
\setbeamertemplate{footline}
{
  \leavevmode%
  \hbox{%
  \begin{beamercolorbox}[wd=.4\paperwidth,ht=2.25ex,dp=1ex,center]{author in head/foot}%
    \usebeamerfont{author in head/foot}\insertshortauthor
  \end{beamercolorbox}%
  \begin{beamercolorbox}[wd=.6\paperwidth,ht=2.25ex,dp=1ex,center]{title in head/foot}%
    \usebeamerfont{title in head/foot}\insertshorttitle\hspace*{3em}
    \insertframenumber{} / \inserttotalframenumber\hspace*{1ex}
  \end{beamercolorbox}}%
  \vskip0pt%
}
\makeatletter
\setbeamertemplate{navigation symbols}{}

\colorlet{LightSpringGreen}{White!70!SpringGreen}
 \usepackage{transparent}
 \newcommand{\semitransp}[2][35]{\color{fg!#1}#2}
\usepackage[T1]{fontenc}	 % For correct hyphenation and T1 encoding
\usepackage{lmodern} % Default font: latin modern font
%\usepackage{fourier} % Alternative font: utopia
%\usepackage{charter} % Alternative font: low-resolution roman font
\renewcommand{\familydefault}{\sfdefault} % Sans serif - this may need to be commented to see the alternative fonts

\usepackage[english]{babel}
\usepackage[utf8x]{inputenc}
\usepackage{xcolor}
\usepackage{listings}

\lstset
{
    language=[LaTeX]TeX,
    breaklines=true,
    basicstyle=\tt\scriptsize,
    %commentstyle=\color{green}
    keywordstyle=\color{blue},
    %stringstyle=\color{black}
    identifierstyle=\color{magenta},
}


\AtBeginSection[]
{
  \begin{frame}<beamer>
    \frametitle{Outline}
    \tableofcontents[currentsection, hideothersubsections]
  \end{frame}
}





\usepackage{listings}
\usepackage{setspace}
\definecolor{Code}{rgb}{0,0,0}
\definecolor{Decorators}{rgb}{0.5,0.5,0.5}
\definecolor{Numbers}{rgb}{0.5,0,0}
\definecolor{MatchingBrackets}{rgb}{0.25,0.5,0.5}
\definecolor{Keywords}{rgb}{0,0,1}
\definecolor{self}{rgb}{0,0,0}
\definecolor{Strings}{rgb}{0,0.63,0}
\definecolor{Comments}{rgb}{0,0.63,1}
\definecolor{Backquotes}{rgb}{0,0,0}
\definecolor{Classname}{rgb}{0,0,0}
\definecolor{FunctionName}{rgb}{0,0,0}
\definecolor{Operators}{rgb}{0,0,0}
\definecolor{Background}{rgb}{0.98,0.98,0.98}
\lstdefinelanguage{Python}{
	numbers=left,
	numberstyle=\footnotesize,
	numbersep=1em,
	xleftmargin=1em,
	framextopmargin=2em,
	framexbottommargin=2em,
	showspaces=false,
	showtabs=false,
	showstringspaces=false,
	frame=l,
	tabsize=4,
	% Basic
	basicstyle=\ttfamily\small\setstretch{1},
	backgroundcolor=\color{Background},
	% Comments
	commentstyle=\color{Comments}\slshape,
	% Strings
	stringstyle=\color{Strings},
	morecomment=[s][\color{Strings}]{"""}{"""},
	morecomment=[s][\color{Strings}]{'''}{'''},
	% keywords
	morekeywords={import,from,class,def,for,while,if,is,in,elif,else,not,and,or,print,break,continue,return,True,False,None,access,as,,del,except,exec,finally,global,import,lambda,pass,print,raise,try,assert},
	keywordstyle={\color{Keywords}\bfseries},
	% additional keywords
	morekeywords={[2]@invariant,pylab,numpy,np,scipy},
	keywordstyle={[2]\color{Decorators}\slshape},
	emph={self},
	emphstyle={\color{self}\slshape},
	%
}

\usepackage{import}

\newcommand{\uline}[1]{\rule[0pt]{#1}{0.4pt}}

\usepackage[para]{footmisc}

%------------------------------------------------
% Colors

%------------------------------------------------

%------------------------------------------------
%------------------------------------------------

%------------------------------------------------
% Fonts
\usepackage[T1]{fontenc}	 % For correct hyphenation and T1 encoding
\usepackage{lmodern} % Default font: latin modern font
%\usepackage{fourier} % Alternative font: utopia
%\usepackage{charter} % Alternative font: low-resolution roman font
\renewcommand{\familydefault}{\sfdefault} % Sans serif - this may need to be commented to see the alternative fonts
%------------------------------------------------

%------------------------------------------------
% Various required packages
\usepackage{amsthm} % Required for theorem environments
\usepackage{bm} % Required for bold math symbols (used in the footer of the slides)
\usepackage{graphicx} % Required for including images in figures
\usepackage{tikz} % Required for colored boxes
\usepackage{booktabs} % Required for horizontal rules in tables
\usepackage{multicol} % Required for creating multiple columns in slides
\usepackage{lastpage} % For printing the total number of pages at the bottom of each slide
\usepackage[english]{babel} % Document language - required for customizing section titles
\usepackage{microtype} % Better typography
\usepackage{tocstyle} % Required for customizing the table of contents


%\newtheorem{defi}{Definition}[section]
%\newtheorem{exmp}{Exercise}[section] %Label for examples
\newtheorem{remark}[theorem]{Remark} % Label for remarks
\newtheorem{algorithm}[theorem]{Algorithm} % Label for algorithms
\makeatletter % Correct qed adjustment
%------------------------------------------------

%------------------------------------------------
% The code for the box which can be used to highlight an element of a slide (such as a theorem)
\newcommand*{\mybox}[2]{ % The box takes two arguments: width and content
\par\noindent
\begin{tikzpicture}[mynodestyle/.style={rectangle,draw=Black,thick,inner sep=1.5mm, text justified,top color=white,bottom color=white,above}]\node[mynodestyle,at={(0.5*#1+2mm+0.4pt,0)}]{ % Box formatting
\begin{minipage}[t]{#1}
#2
\end{minipage}
};
\end{tikzpicture}
\par\vspace{-1.3em}}
%------------------------------------------------

%------------------------------------------------
% MODIFICATIONS BY JUSTIN STEVENS
%------------------------------------------------

\usepackage[nodayofweek,level]{datetime}
\usepackage{caption}
\usepackage{subcaption}
\usepackage{hyperref}
\newcommand{\pmid}{\mid\!\mid}
\usepackage{seqsplit}
\usepackage{amsfonts}
\usepackage{float} %use H to force it in place
\usepackage{amssymb} %for nmid
%\usepackage{enumitem} %for itemized lists with stars
\usepackage{amsmath}
\DeclareMathOperator{\lcm}{lcm}
%\usepackage{epigraph}
\usepackage{csquotes}
\usepackage{relsize}
\newcommand{\x}{\color{red}X\color{black}}

\usetikzlibrary{tikzmark}

\usepackage{textcomp}
\newcommand{\ballgolftikz}[1]{%
	\foreach \i  in {0,...,\number\numexpr#1 - 1\relax}{% 
		\pgfmathsetmacro\k{\i*sqrt(3)/2}
		\begin{scope}[shift={(\i*.5 cm,\k cm)}]
			\foreach \t in {1,...,\number\numexpr #1-\i\relax}{
				\shade[ball color= gray] (\t,0) circle (.5cm);}
	\end{scope}}
}  

\usepackage{pifont}
\usepackage{marginnote}
\reversemarginpar
\newcommand{\prechili}{\vspace*{1.2em}\hspace*{1.0em}}
\newcommand{\nochili}{\hspace*{3.8em}}
\newcommand{\chili}{\includegraphics[width=1.0em]{images/chili.png}}
\newcommand{\gim}{\marginnote{\chili}}
\newcommand{\yod}{\marginnote{\chili\chili}}
\newcommand{\kurumi}{\marginnote{\chili\chili\chili}}
\newcommand{\pencil}{\prechili\marginnote{\bfseries\ding{48}}}
\newcommand{\defi}{{\bfseries\color{ForestGreen}Definition. }}

\usepackage{mathtools}

\newenvironment{polyalign}[1][9]
{\array{c*{#1}{@{}>{{}}c<{{}}@{}c@{}}}}
{\endarray}

\usepackage{tabularx}
\usepackage{bm}
\usepackage{mwe}% provides example images (when installed)
\newcommand\measureISpecification{6ex}% not defined in mwe
\newcommand{\ctab}[1]{\raisebox{\dimexpr \measureISpecification/2 -.748ex}{#1}}% vertically centers numbers

\usetikzlibrary{tikzmark}

\usetikzlibrary{arrows} 

\usepgflibrary{fpu}
\usetikzlibrary{positioning}
\usepackage{thmtools}
\theoremstyle{definition}
\declaretheorem[name=\bfseries Problem]{prob}
%\declaretheorem[name=\bfseries Example]{exmp}
\theoremstyle{plain}

\newenvironment{soln}{\begin{myproof}[Solution]}{\end{myproof}}
\newcommand*\circled[1]{\tikz[baseline=(char.base)]{% <---- BEWARE
		\node[shape=circle,draw,inner sep=2pt] (char) {#1};}}
	
\usepackage{comment}
\usepackage{systeme}


\usepackage{scalerel}
\usepackage{stackengine}
\newcommand\showdiv[1]{\overline{\smash{\hstretch{.5}{)}\mkern-3.2mu\hstretch{.5}{)}}#1}}
\newcommand\ph[1]{\textcolor{white}{#1}}


\usepackage{animate}
\usepackage{enumerate}
\setbeamertemplate{title page}
{
  \vbox{}
  \begingroup
    \centering
    {\usebeamercolor[fg]{titlegraphic}\inserttitlegraphic\par}\vskip1em
    \begin{beamercolorbox}[sep=8pt,center]{title}
      \usebeamerfont{title}\inserttitle\par%
      \ifx\insertsubtitle\@empty%
      \else%
        \vskip0.25em%
        {\usebeamerfont{subtitle}\usebeamercolor[fg]{subtitle}\insertsubtitle\par}%
      \fi%
    \end{beamercolorbox}%
    \vskip1em\par
    \begin{beamercolorbox}[sep=8pt,center]{author}
      \insertauthor
    \end{beamercolorbox}
        \vskip1em\par

  \endgroup
  \vfill
}

\titlegraphic{\includegraphics[height=0.15\textwidth]{../logo.png}}
\title[System of Equations (Lecture 1)]{System of Equations} 
\subtitle{Lecture 1}

\author[Justin Stevens (Star League)]{\large Justin Stevens} % Your name
\date{}


\begin{document}
\begin{frame}[c]
\centering
\titlepage
\end{frame}

\section{Word Problems}
\begin{frame}[c]{Word Problems}
	\centering
	\mybox{0.9\textwidth}{\begin{exmp} Two years ago, Gene was nine times as old as Carol. He is now seven times as old as she is. How many years from now will Gene be five times as old as Carol? (\textit{Source: Mandelbrot})\end{exmp} \begin{exmp} When a bucket is two-thirds full of water, the bucket and water weigh $x$ kilograms. When the bucket is one-half full of water the total weight is $ y$ kilograms. In terms of $x$ and $ y$, what is the total weight in kilograms when the bucket is full of water? (\textit{Source: AMC 12}) \end{exmp}}
\end{frame}

\begin{frame}{Gene and Carol}
		\mybox{0.9\textwidth}{\begin{exmp*} Two years ago, Gene was nine times as old as Carol. He is now seven times as old as she is. How many years from now will Gene be five times as old as Carol?  \end{exmp*}} ~\\
	\begin{soln}
		Let the current ages of Gene and Carol be $g$ and $c$ respectively. Currently, we have $g=7c$. Two years ago, Gene was $g-2$ years old and Carol was $c-2$ years old. Therefore, $g-2=9(c-2)$. We thus have the system of equations: \begin{align*} g&=7c \\ 
		g-2&=9(c-2). \end{align*}
		Substituting the first equation into the second gives $$7c-2=9(c-2)\implies 7c-2=9c-18\implies 2c=16\implies c=8.$$
		Similarly, $g=7\cdot 8=56$. Let the amount of time be $t$. Therefore, $$56+t=5\left(8+t\right)\implies 56+t=40+5t\implies t=4\text{ years}.$$
	\end{soln}
\end{frame}

\begin{frame}{Bucket Full of Water}
	\mybox{0.9\textwidth}{\begin{exmp*} When a bucket is two-thirds full of water, the bucket and water weigh $x$ kilograms. When the bucket is one-half full of water the total weight is $ y$ kilograms. In terms of $x$ and $ y$, what is the total weight in kilograms when the bucket is full of water?  \end{exmp*}} ~\\
	
	Let the weight of the bucket without any water be $b$ and the weight of the water be $w$. Then, we have the system of equations \begin{align*} b+2/3w&=x \\
	b+1/2w&=y. \end{align*}
	Subtracting these gives $\frac{1}{6}w=x-y\implies w=6x-6y$. \medskip
	
	Substituting gives $b+\left(3x-3y\right)=y\implies b=4y-3x$. Hence, $$b+w=\left(4y-3x\right)+\left(6x-6y\right)=\boxed{3x-2y}.$$
\end{frame}
\section{Equations in Three Variables}
\begin{frame}[c]{Equations in Three Variables}
	\centering
	\mybox{0.95\textwidth}{
		\begin{exmp} Solve the below system for $x,y, z$:
			$$\systeme{
				x+2y+3z=3,
				2x+5y+7z=22,
				-x+y+z=6}$$ \end{exmp}
		\begin{exmp} Solve the below system for $r, s, t$: 
$$\systeme{
		r+s+t=7,
		2r+3s-5t=11,
		8r+11s-13t=47}$$ \end{exmp}}
\end{frame}

\begin{frame}[c]
	\centering
	\includegraphics[height=0.6\textwidth]{../images/findingx}
\end{frame}

\begin{frame}{$1$ Unique Solution}
		\mybox{0.95\textwidth}{
		\begin{exmp*} Solve the below system for $x,y, z$:
			$$\systeme{
				x+2y+3z=3,
				2x+5y+7z=22,
				-x+y+z=6}$$ \end{exmp*}} ~\\
	Multiplying the first equation by $2$ and subtracting gives $y+z=16$. ~\\
	
Adding the first and third equation gives $3y+4z=9$. ~\\

Solving this yields $(x, y, z)=(10, 55, -39)$. 
\end{frame}

\begin{frame}{$1$ Solution Dependent on $t$}
\mybox{0.95\textwidth}{		\begin{exmp*} Solve the below system for $r, s, t$: 
		$$\systeme{
			r+s+t=7,
			2r+3s-5t=11,
			8r+11s-13t=47}$$ \end{exmp*}} ~\\
Multiply the first equation by $2$ and second equation by $3$ and add these: \begin{align*} 2\left(r+s+t\right)+3\left(2r+3s-5t\right)&=2\cdot 7+3\cdot 11 \\
8r+11s-13t&=47. \end{align*}
This is exactly the third equation! We consider the first two equations. \medskip 

We find that $(r, s, t)=(10-8t, -3+7t, t)$. Substitute and verify!		
\end{frame}

\begin{frame}[c]{Follow Up}
	\centering
	\mybox{0.95\textwidth}{	\begin{exmp} Solve the below system for $x,y, z$: 
			$$\systeme{
				x-y+z=12,
				2x+3y-4z=5,
				7x+8y-11z=42}$$ \end{exmp}	\begin{exmp} Solve the below system for $r, s, t$: 
			$$\systeme{
				r+2s+3t=7,
				2r+4s+6t=14,
				3r+6s+9t=21}$$ \end{exmp}} 
\end{frame}

\begin{frame}{No Solutions}
	\mybox{0.95\textwidth}{	\begin{exmp*} Solve the below system for $x,y, z$: 
		$$\systeme{
			x-y+z=12,
			2x+3y-4z=5,
			7x+8y-11z=42}$$ \end{exmp*}} ~\\
		
		Multiplying the second equation by $3$ and add this to the first:
		\begin{align*} 3\left(2x+3y-4z\right)+\left(x-y+z\right)&=3\cdot 5+12 \\
					7x+8y-11z &=27. \end{align*}
		Hence, this system has no solutions.
\end{frame}

\begin{frame}{$2$ Basis Solutions}
	\mybox{0.95\textwidth}{\begin{exmp*} Solve the below system for $r, s, t$: 
			$$\systeme{
				r+2s+3t=7,
				2r+4s+6t=14,
				3r+6s+9t=21}$$ \end{exmp*}} ~\\
			
Multiplying the first equation by $2$ gives the second equation. \medskip \pause

Multiplying the first equation by $3$ gives the third equation. \medskip \pause

Hence, the solution is $(r, s, t)=(7-2s-3t, s, t)$. 
\end{frame}

\section{Symmetry}

\begin{frame}[c]{Symmetry Problems}
	\centering
	\mybox{0.95\textwidth}{\begin{exmp} Solve the following system of equations in $w, x, y, z$: \begin{align*} 3w+x+y+z &=20 \\ w+3x+y+z&=6 \\ w+x+3y+z&=44 \\ w+x+y+3z&=2. \end{align*} \end{exmp}
		\begin{exmp} Justin, Lazar, and Daniel are each thinking of a positive number. The product of Justin's and Lazar's is $27$. The product of Lazar's and Daniel's is $72$. The product of Justin's and Daniel's is $6$. Find each person's number. \end{exmp} \begin{exmp} If $ x$, $ y$, and $ z$ are positive numbers satisfying $ x + \dfrac{1}{y} = 4$, $ y + \dfrac{1}{z} = 1$, and $ z + \dfrac{1}{x} = 7/3$ then what is $xyz$? \end{exmp}}
\end{frame}

\begin{frame}{System in $4$ variables}
	\mybox{0.95\textwidth}{\begin{exmp*} Solve the following system of equations in $w, x, y, z$: \begin{align*} 3w+x+y+z &=20 \\ w+3x+y+z&=6 \\ w+x+3y+z&=44 \\ w+x+y+3z&=2. \end{align*} \end{exmp*}} ~\\
	Adding these equations gives $$6\left(w+x+y+z\right)=72\implies w+x+y+z=12.$$ This yields the solution $(w, x, y, z)=(4, -3, 16, -5)$. 
\end{frame}

\begin{frame}{What numbers are they thinking of?}
	\mybox{0.95\textwidth}{		\begin{exmp*} Justin, Lazar, and Daniel are each thinking of a positive number. The product of Justin's and Lazar's is $27$. The product of Lazar's and Daniel's is $72$. The product of Justin's and Daniel's is $6$. Find each person's number. \end{exmp*}} ~\\
	We have the system of equations \begin{align*} jl &=27 \\ ld&=72 \\ jd&=6. \end{align*}
	Multiplying gives $\left(jl d\right)^2=27\cdot 72\cdot 6=3^6\cdot 2^4\implies jl d=108$. \medskip
	
	Therefore, $(j, l, d)=(\frac32, 18, 4)$.
	
\end{frame}



\begin{frame}{More Symmetry}
	\mybox{0.95\textwidth}{\begin{exmp*} If $ x$, $ y$, and $ z$ are positive numbers satisfying $ x + \dfrac{1}{y} = 4$, $ y + \dfrac{1}{z} = 1$, and $ z + \dfrac{1}{x} = 7/3$ then what is $xyz$? \end{exmp*}} ~\\
	
We want to find the product $xyz$, therefore, we multiply the three equations: \small \begin{align*}\left(x+\frac{1}{y}\right)\left(y+\frac{1}{z}\right)\left(z+\frac{1}{x}\right)&=xyz+\frac{xy}{x}+\frac{xz}{z}+\frac{x}{zx}+\frac{yz}{y}+\frac{y}{yx}+\frac{z}{yz}+\frac{1}{xyz} \\ &= xyz+\left(x+y+z\right)+\left(\frac{1}{x}+\frac{1}{y}+\frac{1}{z}\right)+\frac{1}{xyz} \\ &= 4\cdot 1 \cdot \frac73=\frac{28}{3}. \end{align*}  \normalsize

Adding the original three equations: $$\left(x+\frac{1}{y}\right)+\left(y+\frac{1}{z}\right)+\left(z+\frac{1}{x}\right)=4+1+\frac{7}{3}=\frac{22}{3}$$

How do we now find $xyz$? 

\end{frame}

\begin{frame}{More Symmetry II}
	\mybox{0.95\textwidth}{\begin{exmp*} If $ x$, $ y$, and $ z$ are positive numbers satisfying $ x + \dfrac{1}{y} = 4$, $ y + \dfrac{1}{z} = 1$, and $ z + \dfrac{1}{x} = 7/3$ then what is $xyz$? \end{exmp*}} ~\\
Substituting this into the first equation gives us $$xyz+\frac{22}{3}+\frac{1}{xyz}=\frac{28}{3}\implies xyz+\frac{1}{xyz}=2.$$

Finally, multiplying by $xyz$ and re-arranging gives $$(xyz-1)^2=0\implies xyz=\boxed{1}.$$
\end{frame}


\begin{frame}{Final Symmetry Problems}
	\mybox{0.95\textwidth}{\begin{exmp} Let $x,y,z$ be positive real numbers satisfying the simultaneous equations \begin{align*} x(y^2+yz+z^2) &= 3y+10z \\ y(z^2+zx+x^2) &= 21z+24x \\ z(x^2+xy+y^2) &= 7x+28y. \end{align*}  
			Find $xy+yz+zx$. \textit{(Source: 2014 Purple Comet)}  \end{exmp} \begin{exmp} Let $a,b,$ and $c$ be non-zero real numbers such that $$\frac{ab}{a+b}=3,\: \frac{bc}{b+c}=4,\: \text{and } \frac{ca}{c+a}=5.$$ Compute the value of $\displaystyle \frac{abc}{ab+bc+ca}$. \textit{(Source: 2012 Purple Comet)} \end{exmp} }
\end{frame}

\begin{frame}{2014 Purple Comet I}
	We note that the equations on the left hand side are symmetric. We therefore think to \textbf{sum} up the equations. I claim that when we do so, the left hand side becomes $(x+y+z)(xy+xz+yz)$. \color{red} Why? \color{black} \medskip
	
	Expand out, group, and colour the $xyz$ terms: \small \begin{eqnarray*} (x+y+z)(xy+xz+yz)=\color{red}\left(x^2y+ x^2z\right)\color{black}+\color{ForestGreen}\left(y^2x+y^2z\right)\color{black}+\color{blue}\left(z^2y+z^2x\right)\color{black}+3xyz \end{eqnarray*} \normalsize We distribute and colour the other terms: \begin{eqnarray*} x\left(y^2+yz+z^2\right)&=&\color{ForestGreen}y^2x\color{black}+xyz+\color{blue}z^2x \\ y\left(z^2+zx+x^2\right)&=&\color{blue}z^2y\color{black}+xyz+\color{red}x^2y \\ z\left(x^2+xy+y^2\right)&=&\color{red}x^2z\color{black}+xyz+\color{ForestGreen}y^2z. \end{eqnarray*}
	
\end{frame}

\begin{frame}{2014 Purple Comet II}
	
	Therefore, by proof by colouring, we have showed that summing up the left hand side gives $(x+y+z)(xy+xz+yz)$. Summing up the right hand side gives $$\left(3y+10z\right)+\left(21z+24x\right)+\left(7x+28y\right)=31\left(x+y+z\right).$$ We equate the two of them, and divide through by $x+y+z$ (since it is positive): $$(x+y+z)(xy+xz+yz)=31(x+y+z)\implies xy+xz+yz=\boxed{31}.$$ 
	
	How did I think of the factorization? The motivation behind it came from the problem statement asking to find $xy+yz+zx$!
\end{frame}

\begin{frame}{Fractional System}
	\mybox{0.95\textwidth}{\begin{exmp*} Let $a,b,$ and $c$ be non-zero real numbers such that $$\frac{ab}{a+b}=3,\: \frac{bc}{b+c}=4,\: \text{and } \frac{ca}{c+a}=5.$$ Compute the value of $\displaystyle \frac{abc}{ab+bc+ca}$. \textit{(Source: 2012 Purple Comet)} \end{exmp*}} ~\\
Let's take the \textbf{reciprocal} of all of the equations: $$\frac{a+b}{ab}=\frac13\:,\:\: \frac{b+c}{bc}=\frac14\:,\:\: \text{and } \frac{c+a}{ca}=\frac15.$$ 
	Note that we can simplify each of these expressions! For instance, $\frac{a+b}{ab}=\frac{1}{b}+\frac{1}{a}$. Then, the system of equations becomes $$\frac{1}{b}+\frac{1}{a}=\frac13\:,\:\: \frac{1}{c}+\frac{1}{b}=\frac14\:,\:\: \text{and } \frac{1}{a}+\frac{1}{c}=\frac15.$$ 
\end{frame}

\begin{frame}{Fractional System II}
	
	We continue by taking the reciprocal of the desired expression, $\frac{abc}{ab+bc+ca}$: $$\frac{ab+bc+ca}{abc}=\frac{1}{c}+\frac{1}{a}+\frac{1}{b}.$$ Aha! We've turned this into something we know how to work with. We sum up the three equations from the previous slide to get $$2\left(\frac1a+\frac1b+\frac1c\right)=\frac13+\frac14+\frac15=\frac{47}{60}.$$ Therefore, $\displaystyle \frac1a+\frac1b+\frac1c=\frac{47}{120}$. We have to take the reciprocal, however, to get $$\frac{abc}{ab+bc+ca}=\frac{1}{\frac1a+\frac1b+\frac1c}=\boxed{\frac{120}{47}}.$$
	
\end{frame}
	



\begin{frame}[c]{Substitution Problem}
	\centering
	\mybox{0.95\textwidth}{\begin{exmp}[HMMT 2013] Let $x$ and $y$ be real numbers with $x>y$. Find $x$ if $$x^2y^2+x^2+y^2+2xy=40 \text{ and }xy+x+y=8.$$ \textit{(Source: 2013 HMMT)}\end{exmp}}
\end{frame}

\begin{frame}{2013 HMMT}
		\mybox{0.95\textwidth}{\begin{exmp*}[HMMT 2013] Let $x$ and $y$ be real numbers with $x>y$. Find $x$ if $$x^2y^2+x^2+y^2+2xy=40 \text{ and }xy+x+y=8.$$ \end{exmp*}} ~\\

We substitute $a=xy$ and $b=x+y$. The system then becomes: $$a^2+b^2=40\:,\: a+b=8.$$  
	We square the second equation to arrive at $(a+b)^2=a^2+2ab+b^2=64$.  \medskip
	
	Subtracting from the first equation: $2ab=64-40=24\implies ab=12.$ \medskip
	
	Solving this system, we now see that $\color{blue} (a,b)=(2,6), (6,2)$.
\end{frame}

\begin{frame}{2013 HMMT II}
			\mybox{0.95\textwidth}{\begin{exmp*}[HMMT 2013] Let $x$ and $y$ be real numbers with $x>y$. Find $x$ if $$x^2y^2+x^2+y^2+2xy=40 \text{ and }xy+x+y=8.$$ \end{exmp*}} ~\\
	If $(a,b)=(6,2)$, then we have $xy=6$ and $x+y=2$. We use the identity $$(z-x)(z-y)=z^2-z(x+y)+xy.$$ This case gives the quadratic $z^2-2z+6$ with discriminant $\Delta=2^2-4\cdot 6=-20$. Hence, there are no positive real solutions. \medskip
	
	On the other hand, when $(a,b)=(2,6)$, we have $xy=2$ and $x+y=6$. The quadratic for this case is $z^2-6z+2$. Completing the square: $$z^2-6z+2=(z-3)^2-7\implies z=3\pm \sqrt{7}.$$ Since $x>y$, we have $x=\boxed{3+\sqrt{7}}$.  
\end{frame}
\end{document}







\clearpage


\end{document}



\begin{frame}{A Third Funky System}
\mybox{0.95\textwidth}{\begin{exmp*} Solve for $r, s, t$: 
		$$\systeme{
			
\end{document}
	\mybox{0.95\textwidth}{\begin{exmp} Solve the following system of equations in $a,b,c$: \begin{align*}7a+5b+c &= 11 \\ a+4b+6c &= 18 \\ 2a+b+3c &= 21. \end{align*}\end{exmp}
	\begin{exmp} Let $x,y,z$ be positive real numbers satisfying  \begin{align*} x(y^2+yz+z^2) &= 3y+10z \\ y(z^2+zx+x^2) &= 21z+24x \\ z(x^2+xy+y^2) &= 7x+28y. \end{align*}  
		Find $xy+yz+zx$. \footnote{Source: 2014 Purple Comet} \end{exmp}}
	
\end{frame}

\end{document}