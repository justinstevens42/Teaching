% !TeX document-id = {9dc4049a-302f-49cf-bd6d-ab733465c3ce}
% !TeX TXS-program:compile = txs:///pdflatex/[--shell-escape]
\documentclass[xcolor=dvipsnames, fontsize=11pt, % Font size
pagesize, % Write page size to dvi or pdf
parskip=half-, t]{beamer}

%
% Choose how your presentation looks.
%
% For more themes, color themes and font themes, see:
% http://deic.uab.es/~iblanes/beamer_gallery/index_by_theme.html
%
\mode<presentation>
  \usetheme{Madrid}      % or try Darmstadt, Madrid, Warsaw, ...
  \usecolortheme{beaver} % or try albatross, beaver, crane, ...
  %\usefonttheme{serif}  % or try serif, structurebold, ...
  \setbeamertemplate{navigation symbols}{}
  \setbeamertemplate{caption}[numbered]
  
\newcounter{example}
\newenvironment<>{exmp}[1][]{%
    \refstepcounter{example}\par\medskip
  \alert{\textbf{Example~\theexample.} }}{}
  
\newenvironment<>{thm}[1][]{%
 \par\medskip
\textbf{\textcolor{MidnightBlue}{\sffamily Theorem.  }}}{}
\newenvironment<>{exmp*}[1][]{%
\par\medskip
  \alert{\textbf{Example.} }}{}
  
\makeatletter
\newenvironment<>{myproof}[1][\proofname]{%
  \par
  \def\insertproofname{#1.}%
  \pushQED{\qed}
  \textit{\insertproofname}  }
{}
\makeatother

\setbeamertemplate{headline}{}
\setbeamercovered{transparent}

\usepackage{etoolbox}
\makeatletter
\patchcmd{\beamer@continueautobreak}{\frametitle}{\beamer@gobbleoptional}{}{\errmessage{failed to patch}}
\patchcmd{\beamer@continueautobreak}{\framesubtitle}{\beamer@gobbleoptional}{}{\errmessage{failed to patch}}
\makeatother

\makeatother
\setbeamertemplate{footline}
{
  \leavevmode%
  \hbox{%
  \begin{beamercolorbox}[wd=.4\paperwidth,ht=2.25ex,dp=1ex,center]{author in head/foot}%
    \usebeamerfont{author in head/foot}\insertshortauthor
  \end{beamercolorbox}%
  \begin{beamercolorbox}[wd=.6\paperwidth,ht=2.25ex,dp=1ex,center]{title in head/foot}%
    \usebeamerfont{title in head/foot}\insertshorttitle\hspace*{3em}
    \insertframenumber{} / \inserttotalframenumber\hspace*{1ex}
  \end{beamercolorbox}}%
  \vskip0pt%
}
\makeatletter
\setbeamertemplate{navigation symbols}{}

\colorlet{LightSpringGreen}{White!70!SpringGreen}
 \usepackage{transparent}
 \newcommand{\semitransp}[2][35]{\color{fg!#1}#2}
\usepackage[T1]{fontenc}	 % For correct hyphenation and T1 encoding
\usepackage{lmodern} % Default font: latin modern font
%\usepackage{fourier} % Alternative font: utopia
%\usepackage{charter} % Alternative font: low-resolution roman font
\renewcommand{\familydefault}{\sfdefault} % Sans serif - this may need to be commented to see the alternative fonts

\usepackage[english]{babel}
\usepackage[utf8x]{inputenc}
\usepackage{xcolor}
\usepackage{listings}

\lstset
{
    language=[LaTeX]TeX,
    breaklines=true,
    basicstyle=\tt\scriptsize,
    %commentstyle=\color{green}
    keywordstyle=\color{blue},
    %stringstyle=\color{black}
    identifierstyle=\color{magenta},
}


\AtBeginSection[]
{
  \begin{frame}<beamer>
    \frametitle{Outline}
    \tableofcontents[currentsection, hideothersubsections]
  \end{frame}
}





\usepackage{listings}
\usepackage{setspace}
\definecolor{Code}{rgb}{0,0,0}
\definecolor{Decorators}{rgb}{0.5,0.5,0.5}
\definecolor{Numbers}{rgb}{0.5,0,0}
\definecolor{MatchingBrackets}{rgb}{0.25,0.5,0.5}
\definecolor{Keywords}{rgb}{0,0,1}
\definecolor{self}{rgb}{0,0,0}
\definecolor{Strings}{rgb}{0,0.63,0}
\definecolor{Comments}{rgb}{0,0.63,1}
\definecolor{Backquotes}{rgb}{0,0,0}
\definecolor{Classname}{rgb}{0,0,0}
\definecolor{FunctionName}{rgb}{0,0,0}
\definecolor{Operators}{rgb}{0,0,0}
\definecolor{Background}{rgb}{0.98,0.98,0.98}
\lstdefinelanguage{Python}{
	numbers=left,
	numberstyle=\footnotesize,
	numbersep=1em,
	xleftmargin=1em,
	framextopmargin=2em,
	framexbottommargin=2em,
	showspaces=false,
	showtabs=false,
	showstringspaces=false,
	frame=l,
	tabsize=4,
	% Basic
	basicstyle=\ttfamily\small\setstretch{1},
	backgroundcolor=\color{Background},
	% Comments
	commentstyle=\color{Comments}\slshape,
	% Strings
	stringstyle=\color{Strings},
	morecomment=[s][\color{Strings}]{"""}{"""},
	morecomment=[s][\color{Strings}]{'''}{'''},
	% keywords
	morekeywords={import,from,class,def,for,while,if,is,in,elif,else,not,and,or,print,break,continue,return,True,False,None,access,as,,del,except,exec,finally,global,import,lambda,pass,print,raise,try,assert},
	keywordstyle={\color{Keywords}\bfseries},
	% additional keywords
	morekeywords={[2]@invariant,pylab,numpy,np,scipy},
	keywordstyle={[2]\color{Decorators}\slshape},
	emph={self},
	emphstyle={\color{self}\slshape},
	%
}

\usepackage{import}

\newcommand{\uline}[1]{\rule[0pt]{#1}{0.4pt}}

\usepackage[para]{footmisc}

%------------------------------------------------
% Colors

%------------------------------------------------

%------------------------------------------------
%------------------------------------------------

%------------------------------------------------
% Fonts
\usepackage[T1]{fontenc}	 % For correct hyphenation and T1 encoding
\usepackage{lmodern} % Default font: latin modern font
%\usepackage{fourier} % Alternative font: utopia
%\usepackage{charter} % Alternative font: low-resolution roman font
\renewcommand{\familydefault}{\sfdefault} % Sans serif - this may need to be commented to see the alternative fonts
%------------------------------------------------

%------------------------------------------------
% Various required packages
\usepackage{amsthm} % Required for theorem environments
\usepackage{bm} % Required for bold math symbols (used in the footer of the slides)
\usepackage{graphicx} % Required for including images in figures
\usepackage{tikz} % Required for colored boxes
\usepackage{booktabs} % Required for horizontal rules in tables
\usepackage{multicol} % Required for creating multiple columns in slides
\usepackage{lastpage} % For printing the total number of pages at the bottom of each slide
\usepackage[english]{babel} % Document language - required for customizing section titles
\usepackage{microtype} % Better typography
\usepackage{tocstyle} % Required for customizing the table of contents


%\newtheorem{defi}{Definition}[section]
%\newtheorem{exmp}{Exercise}[section] %Label for examples
\newtheorem{remark}[theorem]{Remark} % Label for remarks
\newtheorem{algorithm}[theorem]{Algorithm} % Label for algorithms
\makeatletter % Correct qed adjustment
%------------------------------------------------

%------------------------------------------------
% The code for the box which can be used to highlight an element of a slide (such as a theorem)
\newcommand*{\mybox}[2]{ % The box takes two arguments: width and content
\par\noindent
\begin{tikzpicture}[mynodestyle/.style={rectangle,draw=Black,thick,inner sep=1.5mm, text justified,top color=white,bottom color=white,above}]\node[mynodestyle,at={(0.5*#1+2mm+0.4pt,0)}]{ % Box formatting
\begin{minipage}[t]{#1}
#2
\end{minipage}
};
\end{tikzpicture}
\par\vspace{-1.3em}}
%------------------------------------------------

%------------------------------------------------
% MODIFICATIONS BY JUSTIN STEVENS
%------------------------------------------------

\usepackage[nodayofweek,level]{datetime}
\usepackage{caption}
\usepackage{subcaption}
\usepackage{hyperref}
\newcommand{\pmid}{\mid\!\mid}
\usepackage{seqsplit}
\usepackage{amsfonts}
\usepackage{float} %use H to force it in place
\usepackage{amssymb} %for nmid
%\usepackage{enumitem} %for itemized lists with stars
\usepackage{amsmath}
\DeclareMathOperator{\lcm}{lcm}
%\usepackage{epigraph}
\usepackage{csquotes}
\usepackage{relsize}
\newcommand{\x}{\color{red}X\color{black}}

\usetikzlibrary{tikzmark}

\usepackage{textcomp}
\newcommand{\ballgolftikz}[1]{%
	\foreach \i  in {0,...,\number\numexpr#1 - 1\relax}{% 
		\pgfmathsetmacro\k{\i*sqrt(3)/2}
		\begin{scope}[shift={(\i*.5 cm,\k cm)}]
			\foreach \t in {1,...,\number\numexpr #1-\i\relax}{
				\shade[ball color= gray] (\t,0) circle (.5cm);}
	\end{scope}}
}  

\usepackage{pifont}
\usepackage{marginnote}
\reversemarginpar
\newcommand{\prechili}{\vspace*{1.2em}\hspace*{1.0em}}
\newcommand{\nochili}{\hspace*{3.8em}}
\newcommand{\chili}{\includegraphics[width=1.0em]{images/chili.png}}
\newcommand{\gim}{\marginnote{\chili}}
\newcommand{\yod}{\marginnote{\chili\chili}}
\newcommand{\kurumi}{\marginnote{\chili\chili\chili}}
\newcommand{\pencil}{\prechili\marginnote{\bfseries\ding{48}}}
\newcommand{\defi}{{\bfseries\color{ForestGreen}Definition. }}

\usepackage{mathtools}

\newenvironment{polyalign}[1][9]
{\array{c*{#1}{@{}>{{}}c<{{}}@{}c@{}}}}
{\endarray}

\usepackage{tabularx}
\usepackage{bm}
\usepackage{mwe}% provides example images (when installed)
\newcommand\measureISpecification{6ex}% not defined in mwe
\newcommand{\ctab}[1]{\raisebox{\dimexpr \measureISpecification/2 -.748ex}{#1}}% vertically centers numbers

\usetikzlibrary{tikzmark}

\usetikzlibrary{arrows} 

\usepgflibrary{fpu}
\usetikzlibrary{positioning}
\usepackage{thmtools}
\theoremstyle{definition}
\declaretheorem[name=\bfseries Problem]{prob}
%\declaretheorem[name=\bfseries Example]{exmp}
\theoremstyle{plain}

\newenvironment{soln}{\begin{myproof}[Solution]}{\end{myproof}}
\newcommand*\circled[1]{\tikz[baseline=(char.base)]{% <---- BEWARE
		\node[shape=circle,draw,inner sep=2pt] (char) {#1};}}
	
\usepackage{comment}
\usepackage{systeme}


\usepackage{scalerel}
\usepackage{stackengine}
\newcommand\showdiv[1]{\overline{\smash{\hstretch{.5}{)}\mkern-3.2mu\hstretch{.5}{)}}#1}}
\newcommand\ph[1]{\textcolor{white}{#1}}


\usepackage{animate}
\usepackage{enumerate}
\setbeamertemplate{title page}
{
  \vbox{}
  \begingroup
    \centering
    {\usebeamercolor[fg]{titlegraphic}\inserttitlegraphic\par}\vskip1em
    \begin{beamercolorbox}[sep=8pt,center]{title}
      \usebeamerfont{title}\inserttitle\par%
      \ifx\insertsubtitle\@empty%
      \else%
        \vskip0.25em%
        {\usebeamerfont{subtitle}\usebeamercolor[fg]{subtitle}\insertsubtitle\par}%
      \fi%
    \end{beamercolorbox}%
    \vskip1em\par
    \begin{beamercolorbox}[sep=8pt,center]{author}
      \insertauthor
    \end{beamercolorbox}
        \vskip1em\par

  \endgroup
  \vfill
}

\titlegraphic{\includegraphics[height=0.15\textwidth]{../logo.png}}
\title[Exponents and Radicals (Lecture 3)]{Exponents and Radicals}
\subtitle{Lecture 3}

\author[Justin Stevens (Star League)]{\large Justin Stevens} % Your name
\date{}


\begin{document}
	\begin{frame}[c]
		\centering
		\titlepage
	\end{frame}



\begin{frame}[c]{Exponent Mayhem}
	\centering
	\mybox{0.95\textwidth}{ \begin{exmp} Let $A=x+\frac{1}{x}$ and $B=x^2+\frac{1}{x^2}$. Note that $(x+\frac{1}{x})^2=x^2+2+\frac{1}{x^2}$, therefore, $B=A^2-2$. Find formulas for $$C=x^3+\frac{1}{x^3}\:, D=x^4+\frac{1}{x^4}, E=x^5+\frac{1}{x^5}$$ in terms of $A$. \footnote{Source: AoPS Introduction to Algebra} \end{exmp}}

\end{frame}


\begin{frame}{Exponent Mayhem}
	\mybox{0.95\textwidth}{\begin{exmp*} Find formulas for $C=x^3+\frac{1}{x^3}\:, D=x^4+\frac{1}{x^4}, E=x^5+\frac{1}{x^5}$ in terms of $A=x+\frac{1}{x}$. \end{exmp*}} ~\\
	
	To compute $C$, we cube $x+\frac{1}{x}$: \begin{align*} (x+\frac{1}{x})^3&=x^3+3\cdot x^2\cdot \frac{1}{x}+3\cdot x\cdot \frac{1}{x^2}+\frac{1}{x^3} \\ &= x^3+3x+\frac{3}{x}+\frac{1}{x^3}. \end{align*}
	Therefore, substituting $A=x+\frac{1}{x}$ gives \begin{align*} A^3&=x^3+3\left(x+\frac{1}{x}\right)+\frac{1}{x^3} \:=\:C+3A \\ \implies \color{blue} C &= \color{blue} A^3-3A. \end{align*}
\end{frame}

\begin{frame}{Exponent Mayhem II}
	\mybox{0.95\textwidth}{\begin{exmp*} Find formulas for $C=x^3+\frac{1}{x^3}\:, D=x^4+\frac{1}{x^4}, E=x^5+\frac{1}{x^5}$ in terms of $A=x+\frac{1}{x}$. \end{exmp*}} ~\\
	
	There are two methods for finding $D$. One of them involves taking $x+\frac{1}{x}$ to the fourth power. In order to continue with this method, however, I must introduce the binomial theorem and Pascal's triangle.  
	
	\begin{figure}[h]
		\centering\includegraphics[width=0.36\linewidth]{../images/pascal.jpg}
		\caption{Source:  iCoachMath.com}
	\end{figure}
\end{frame}

\begin{frame}{Binomial Theorem}
	
	The binomial theorem states that when we expand $x+y$ to the $n$th power, the coefficients will be the numbers in the $n$th row of Pascal's triangle. For instance, $$(x+y)^4=\textbf{1}x^4+\textbf{4}x^3y+\textbf{6}x^2y^2+\textbf{4}xy^3+\textbf{1}y^4.$$
	The numbers $1, 4, 6, 4, 1$ make up the $4$th row of Pascal's triangle. \medskip 
	
	Furthermore, if you know binomial coefficients, note that $$\binom{4}{0}=1, \binom{4}{1}=4, \binom{4}{2}=6, \binom{4}{3}=4, \binom{4}{4}=1.$$
	
	\mybox{0.8\textwidth}{\begin{thm}[Binomial Expansion] $$(x+y)^n=\displaystyle \sum_{k=0}^{n}\binom{n}{k}x^ky^{n-k}.$$ \end{thm}}
\end{frame}

\begin{frame}{Exponent Mayhem III}
	\mybox{0.95\textwidth}{\begin{exmp*} Find formulas for $C=x^3+\frac{1}{x^3}\:, D=x^4+\frac{1}{x^4}, E=x^5+\frac{1}{x^5}$ in terms of $A=x+\frac{1}{x}$. \end{exmp*}} ~\\
	
	Using the expansion for $(x+y)^4$, we see that \begin{align*} (x+\frac{1}{x})^4&=x^4+4\cdot \left(x^3\cdot \frac{1}{x}\right)+6\cdot \left(x^2\cdot \frac{1}{x^2}\right)+4\cdot \left(x\cdot \frac{1}{x^3}\right)+\frac{1}{x^4} \\ &= \left(x^4+\frac{1}{x^4}\right)+4\left(x^2+\frac{1}{x^2}\right)+6. \end{align*}
	We substitute the formula $B=x^2+\frac{1}{x^2}=A^2-2$ to get: $$\color{blue} D=x^4+\frac{1}{x^4}=A^4-4(A^2-2)-6=A^4-4A^2+2.$$ 
	
\end{frame}

\begin{frame}{Exponent Mayhem IV}
	\mybox{0.95\textwidth}{\begin{exmp*} Find formulas for $C=x^3+\frac{1}{x^3}\:, D=x^4+\frac{1}{x^4}, E=x^5+\frac{1}{x^5}$ in terms of $A=x+\frac{1}{x}$. \end{exmp*}} ~\\
	
	A simpler method exists for computing $D$ without the use of the binomial theorem. Note that if we multiply $A$ by $C$, we get the desired $x^4$ and $\frac{1}{x^4}$ terms: $$AC=\left(x+\frac{1}{x}\right)\left(x^3+\frac{1}{x^3}\right)=x^4+\left(x^2+\frac{1}{x^2}\right)+\frac{1}{x^4}.$$  
	
	From above, we found $C=A^3-3A$. Furthermore, $B=x^2+\frac{1}{x^2}=A^2-2$. Substituting these both in give $$D=x^4+\frac{1}{x^4}=A(A^3-3A)-(A^2-2)=A^4-4A^2+2.$$ Note this matches the answer above.  
	
\end{frame}

\begin{frame}{Exponent Mayhem V}
	\mybox{0.95\textwidth}{\begin{exmp*} Find formulas for $C=x^3+\frac{1}{x^3}\:, D=x^4+\frac{1}{x^4}, E=x^5+\frac{1}{x^5}$ in terms of $A=x+\frac{1}{x}$. \end{exmp*}} ~\\	
	We attempt our new method for computing $E$. Note that if we multiply $A$ by $D$, we get the desired $x^5$ and $\frac{1}{x^5}$ terms: $$AD=\left(x+\frac{1}{x}\right)\left(x^4+\frac{1}{x^4}\right)=x^5+\left(x^3+\frac{1}{x^3}\right)+\frac{1}{x^5}.$$  
	We substitute $D=A^4-4A^2+2$ and $C=x^3+\frac{1}{x^3}=A^3-3A$ into the above equation: $$\color{blue} E=x^5+\frac{1}{x^5}=A\left(A^4-4A^2+2\right)-\left(A^3-3A\right)=A^5-5A^3+5A.$$In general, if $x_n=x^n+\frac{1}{x^n}$, then we can recursively find the next term using the identity $$x_1x_{n-1}=x_n+x_{n-2}\implies x_n=x_1x_{n-1}-x_{n-2}.$$ 
\end{frame}

\begin{frame}{Exponent Mayhem in NIMO}
	The identity above was a key motivator in a 2015 National Internet Math Olympiad (NIMO) challenge problem I cowrote with Evan Chen! \medskip 
	
	\mybox{0.95\textwidth}{\begin{exmp*}(Justin Stevens and Evan Chen)  Let $a$, $b$, $c$ be reals and $p$ be a prime number. Assume that \[ a^n(b+c)+b^n(a+c)+c^n(a+b)\equiv 8\pmod{p} \] for each nonnegative integer $n$. Let $m$ be the remainder when $a^p+b^p+c^p$ is divided by $p$, and $k$ the remainder when $m^p$ is divided by $p^4$. Find the maximum possible value of $k$. \end{exmp*}} ~\\ \medskip
	
	The solution involves finding similar recursion relations with some number theory tricks as well. The answer is $399$; try to figure out why after finishing this course!
\end{frame}

\end{document}
