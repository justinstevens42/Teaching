%Author:  Justin Stevens
%Last Modified:  August 2nd, 2016
\documentclass[12pt,openany]{book}
\setlength{\headheight}{15pt}
\usepackage{amsmath, amsthm, amssymb}
\usepackage{mdframed}
\usepackage{lipsum}
\newmdtheoremenv{thm}{Theorem}[section]
\newmdtheoremenv{exmp}{Example}[section]
\theoremstyle{definition}
\newtheorem{defi}{Definition}[section]
\newenvironment{soln}{\begin{proof}[Solution]}{\end{proof}}
\newtheorem{prob}{Problem}[section]
\newtheorem*{comment}{Comment}
\setcounter{chapter}{1}
\setcounter{section}{0}
\usepackage{titlesec}
\renewcommand*\thesection{\arabic{section}}  %important  
\usepackage{cancel}
\usepackage[margin=3cm]{geometry}
\usepackage{hyperref}
\usepackage{fancyhdr}
\pagestyle{fancy}
\fancyhead{}
\fancyfoot{}
\theoremstyle{definition}
\newtheorem*{case}{Example}
\rhead{Page \thepage}
\newenvironment{dedication}
    {\vspace{6ex}\begin{quotation}\begin{center}\begin{em}}
    {\par\end{em}\end{center}\end{quotation}}
\newcommand{\HRule}{\rule{\linewidth}{0.5mm}} % Defines a new command for the horizontal lines, change thickness here

%----------------------------------------------------------------------------------------
%	PRESENTATION INFORMATION
%----------------------------------------------------------------------------------------

\newcommand*{\mytitle}{Building Blocks of Geometry} % Title
\newcommand*{\runninghead}{AMC Geometry} % Running head displayed on almost all slides
\newcommand*{\myauthor}{Justin Stevens} % Presenters name(s)
\newcommand*{\mydate}{\formatdate{26}{12}{2017}} % Presentation date
\newcommand*{\myuni}{AStar 2016 Winter Math Camp} % University or department

%----------------------------------------------------------------------------------------

\usepackage{asymptote}


\begin{document}
	

%----------------------------------------------------------------------------------------
%	TITLE SLIDE
%----------------------------------------------------------------------------------------

% Title slide - you may have to tweak a few of the numbers if you wish to make changes to the layout
\thispagestyle{empty} % No slide header and footer
\begin{tikzpicture}[remember picture,overlay] % Background box
\node [xshift=\paperwidth/2,yshift=\paperheight/2] at (current page.south west)[rectangle,fill,inner sep=0pt,minimum width=\paperwidth,minimum height=\paperheight/3,top color=mygreen,bottom color=mygreen]{}; % Change the height of the box, its colors and position on the page here
\end{tikzpicture}
% Text within the box
\begin{flushright}
\vspace{0.6cm}
\color{white}\sffamily
{\bfseries\Large\mytitle\par} % Title
\vspace{0.5cm}
\normalsize
\myauthor\par % Author name
\mydate\par % Date
\vfill
\end{flushright}



\clearpage

%----------------------------------------------------------------------------------------
%	PRESENTATION SLIDES
%----------------------------------------------------------------------------------------

\section{Introduction}

\clearpage

\section{Congruences Theorems}
\clearpage
\section{Angles in Polygons}
discussion discussion
\clearpage
\thispagestyle{empty}
\begin{figure}[h!]
	\centering\includegraphics[width=0.9\linewidth]{images/parallel.png}
\end{figure}


\clearpage

\section{Angles in Circles}
\clearpage
\section{Similar Triangles}
\clearpage
\section{Special Points of Triangles}
\clearpage
\section{Power of a Point}
\clearpage
\section{Area of Complex Figures}
\clearpage
\section{Trigonometry}






\end{document}